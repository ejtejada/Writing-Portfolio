%% Author: Edgar Tejada
%% This is 2019-05-19_Research_Review_on_Premeds

\documentclass [12pt]{article}
\title{On the State of Premedical Education and the Self-Selection Criteria of Premedical Students \\\medskip A Literature Review}
\author{Edgar Tejada ([redacted])\\University of California, Merced}
\usepackage[utf8]{inputenc}
\usepackage[backend=biber, bibencoding=utf8, style=mla, citestyle=mla]{biblatex}
%\usepackage[english]{babel} %English only, no greek or math.
\usepackage[greek,english]{babel} %Needed for greek letters
\usepackage[pdfborder={0 0 0.5 [3 3]}, pdftitle={On the State of Premedical Education and the Self-Selection Criteria of Premedical Students}]{hyperref} %Needed for website
\hypersetup{
	colorlinks=true,
}
\usepackage{csquotes}
%\renewcommand{\mktextquote}[6]{#1#2#4#6#3#5} %Will this fix quotes having period on the outside?
\addbibresource{ActualBib.bib}
%To append abstracts to *any* citation style, we will use stack exchange users tmgriffiths and moewe: 
%https://tex.stackexchange.com/a/278419
\DeclareFieldFormat{abstract}{\par\small\textbf{Annotation}: #1}
\renewbibmacro*{finentry}{%
	\iffieldundef{abstract}
	{%
		\finentry
	}
	{%
		\finentrypunct\printfield{abstract}\finentry
	}
}

%
%Margin - 1 inch on all sides
%
\usepackage[letterpaper]{geometry}
\usepackage{times}
\geometry{top=1.0in, bottom=1.0in, left=1.0in, right=1.0in}

%
%Doublespacing
%
\usepackage{setspace}
\doublespacing

%
%Rotating tables (e.g. sideways when too long)
%
\usepackage{rotating}


%
%Fancy-header package to modify header/page numbering (insert last name)
%
\usepackage{fancyhdr}
\pagestyle{fancy}
\lhead{} 
\chead{} 
\rhead{Tejada \thepage} 
\lfoot{} 
\cfoot{} 
\rfoot{} 
\renewcommand{\headrulewidth}{0pt} 
\renewcommand{\footrulewidth}{0pt} 
%To make sure we actually have header 0.5in away from top edge
%12pt is one-sixth of an inch. Subtract this from 0.5in to get headsep value
\setlength\headsep{0.333in}

%Fix Works Cited Heading
\defbibheading{bibliography}{%
	\section*{\centerline{\normalfont {\normalsize Works Cited}}}}%[FIXME] The title still has an oversized double space following it.

%
%(Old) Works cited environment
%(to start, use \begin{workscited...}, each entry preceded by \bibent)
% - from Ryan Alcock's MLA style file
%
\newcommand{\bibent}{\noindent \hangindent 40pt}
\newenvironment{workscited}{\newpage \begin{center} Works Cited 
\end{center}}{\newpage }

%
%Begin document
%
\begin{document}
\begin{flushleft}
%%%%First page name, class, etc
Edgar Tejada\\
Dr. Wilson\\
Writing 117: Writing for the Social Sciences\\
20 May 2019\\


%%%%Title
\begin{center}
	On the State of Premedical Education and \\the Self-Selection Criteria of Premedical Students
\end{center}


%%%%Changes paragraph indentation to 0.5in
\setlength{\parindent}{0.5in}
%%%%Begin body of paper here
Many, aspiring new undergraduate students enter the field of academia with the dream of spring-boarding into the health and medical fields. These ambitious, motivated students want to become the modern world's healers; they want to touch people's lives and fulfill their dreams of becoming doctors. However, many of these students brutally discover that this life-long career choice is both arduous and self-destructive for all but the most dedicated of students. Indeed, the road to doctorhood may be littered with moral pitfalls and the dreams of the super-majority (60\%) that fail to make it (\cite{Olsen-2016} 74). Even so, all of the fault is not on the students, as the current literature shows that colleges and universities do not guide premedical students on the academic, social, and moral battles needed to win entry into medical school.

	%Gameplan: Struggles in Academia
		%They aren't prepared or challeneged
Freshman who want to purse health professions already come from high school with serious disadvantages. On average, they enter college with only a fractional idea of what career to pursue, but in the case of premedical students, without a good road map of what to learn and how to learn it. As stated by Arum, a sociology professor at NYU, {``}Although growing proportions of high school graduates are entering higher education, many are not prepared for college-level work and many others have no clear plan for the future... students proceed to higher education regardless of their academic performance" (\cite{Arum-2011} 9). Still, even for those who select premedical as their path and commit to it, the earliest hurdles are introductory chemistry and biochemistry courses. Both of these are difficult not just due to nature of the subjects, but also because they are often taught poorly. Cooper, the Chemistry department head of Michigan State, noted that although premedical students go through almost two and a half years of chemistry, the inter-relationships between the concepts between classes do not form. She argues that a plurality of these students go through the motions of classes via rote memorization, but come of their time without understanding the applications of chemistry. Instead of building the bonds between the atomic to the biochemical, {``}Students see each step in the process as separate, and are often tested on them separately so that they do not see their overall purpose or value.... students must have relevant prior knowledge, new material must be explicitly connected to that prior knowledge" (\cite{Cooper-2013} 821). While this trend of memorizing over synthesizing seems like it might be due to some negative trait of premeds, further literature shows, that, on the contrary, premeds are trying more than others to understand and grasp the ideas.


		%They want to learn, but aren't given the tools.
One study conducted on the motivations of pre-meds was by Horowitz, a former associate professor of chemistry at Brooklyn College. He performed a study via a series of random selections and interviews, selecting only from the 2009 class that had self-indicated as 'Premed.' In this, Horowitz found that  {``}Only a small number of students (4 of 31) indicated that grades were all that mattered to them" (\cite{Horowitz-2010} 227). To the contrary, further surveying found that many of them desired to master hard STEM courses, and saw that {``}[D]oing well in pre-med courses was not sufficient ... that the real reason for wanting to learn the premed course material was simply in order to do well on the Medical College Admission Test" (\cite{Horowitz-2010} 230). So even though the research shows a plurality of these students desire to intrinsically understand their course material, the structure and learning objects in place by school administration solidly prevent this. For example, Arum criticizes the over reliance on {`}fact' based multiple choice exams and notes how short-lived the learning for these exams can be. He calls for a return to emphasis on free response and open ended questions, and that {``}While [students] may be acquiring subject-specific knowledge ...  many students are not improving their skills in critical thinking, complex reasoning, or writing" (\cite{Arum-2011} 14). All of Arum's learning metrics, which universities currently fail to meet, are also the cornerstones of the Medical College Admission Test, or MCAT, the final challenge that colleges also leave their students to fight without guidance.

		%They aren't given tools needed to conquer the MCAT
The MCAT acts as the ultimate dividing line between the upperclassmen who were planning to apply to medical school and the upperclassmen that actually do apply. However, the question arises if its a metric that universities can  even help their students in? In an hyper competitive applicant pool, {``}[T]here were 47,815 applications to medical schools, and only 20,343 student matriculated. The average GPA for a matriculated student was 3.69" (\cite{Olsen-2016} 74). Thus, one of the only ways to stand against the competition is to do above the 75th percentile of accepted students. This is hardly an easy feat, as the exam is 7.5 hours in length and challenges both depth and breadth of an applicants knowledge. Olsen, a Sociology professor at UC San Diego, describes the nature of the beast, stating that {``}The first two test the applicants' knowledge and use of concepts in biology [and] chemistry... The second two test the applicants' knowledge, use, and critical analysis of behavior and sociocultural determinants" (\cite{Olsen-2016} 72). Olsen uses the new, extended nature of the 2015 MCAT revision to argue that premedical students can stand to gain from focusing on their sociology classes and other 'soft' science lectures; Olsen also strongly claims that these classes can help a student become effective doctors. Olsen makes it a point to praise the added rigor to the reading and critical thinking components as a reminder that if doctors were trapped in only biochemistry classes, they would be unable to humanize their patients (\cite{Olsen-2016} 81). On the contrary, Cooper argues from the perspective of a STEM educator against the MCAT, as {``}Unfortunately, the proposed chemistry learning objectives [in the 2015 MCAT] do not seem to provide a coherent framework with which to work. Rather, these objectives are fragments, disconnected ideas" (\cite{Cooper-2013} 820). As Cooper also argued that the structure of current chemistry intro classes was fundamentally flawed, Cooper concedes that students can either learn chemistry as professors teach it (reductionist), or learn it holistically for the MCAT, but not both. Thus, the literature does not agree on if schools can directly help their students survive the MCAT, but the literature on premeds paints an even bleaker picture on where schools do fail: the social lives of premeds.
	  
	%Gameplan: Struggles in Socializing and Survival ($)
On top of the high likelihood of failure in classes and on the MCAT, the area premeds students need the most help in is being provided avenues to positively socialize with each other. One earlier study on the lack of positive socialization of premeds was conducted by Anaya, a statistician at Indiana University. She took a sample of 425 medical school applicants compiled by UCLA and the American Council of Education, and ran analysis that looked for the strongest factors contributing to medical school acceptance while controlling for factors like gender or race. She noted that time was a luxury for these students, due to the costly {``}Investment of their time and energy in decision making, [which] is guided by the dream of becoming a doctor. The array of decisions and necessary tasks to meet this goal ... indeed requires considerable thought and energy" (\cite{Anaya-2001} 179). Anaya noted that any time spent outside of classes for organizations that were not research, but still helped premeds, was treated as negative factors by medical school admission committees. Even more worrying, the students passing the first round of applications were those that sacrificed to cram more hours for studying. For example, she exclaimed {``}[C]o-curricular activities are negatively associated with student learning as measured by performance on the MCAT. Student involvement in college clubs and organizations (-0.20, p < 0.05) and working for pay (-0.09, p < 0.01) appear to hinder student learning" (\cite{Anaya-2001} 188). This suggests that those who choose, or need, to work and socialize, are looking at performance between 10\% to 20\% worse than their peers who focus only on studying. However, the lack of life-study balance is only made worse by the negative stereotypes premedical students hold for each other.
	
	%The 'gunner' or a-hole sterotypes of premeds
A large array of past and modern literature suggests that premedical students do not trust one another. There is an overarching, inter and intra-cultural belief that premeds are cut-throat to a fault, and that {``}[T]he stereotype persists not only among non-premeds and faculty, who view premeds as less mature, creative, honest, and humane than non-premeds, but also among premeds themselves who see themselves as more anxious, less socially active, and less politically conscious" (\cite{Olsen-2016} 74). Clearly, tt becomes exceedingly difficult for premedical students to form social circles, study groups, or even make friends if being premed acts like a scarlet letter. To make it worse, there appears to be evidence of social pressure from non-premed peers to switch career focus, as Horowitz pointed out in his study (at a male only college) that "A total of 3 students reported that an earlier focus of theirs on grades had been influenced and shifted by exposure to peers outside of JC [the college].... causing them to rethink their goals [regarding med school]" (\cite{Horowitz-2010} 229). Thus, the only premed students that make it to a MCAT score and GPA combo high enough to even be in the running for a medical school application (the top 25th percentile of applicants) are those the forgo romance, forgo clubs and sports, and sacrifice any expendable income they could earn from a job. It stands to reason that the road to doctorhood is a deary, lonely one, with obfuscating curtains made of stress, studying, and long nights. To top it all off, Anaya noted that the metrics by which undergraduate schools could statistically and significantly help their premed populations would be by providing "[S]ensible participation in co-curricular activities, enrollment in greater numbers of science courses, and involvement with faculty (research and out-of class) are suggested by the results of this study" (\cite{Anaya-2001} 189).
Anaya makes clear the unspoken truth, that research and personal connections are one of the strongest gateways into a good medical school. These are not doors that are open if students are crushed by academics, forced to relearn broken fundamentals for the MCAT, or distrusted by their peers and professors. This black curtain engulfing the surviving premedical students directly leads to the lack of (or rejection of) ethical teachings to these premeds.

	%Gameplan: Struggles in Morality in Ethics (Just one paragraph)
	Finally
	
	%Closing Remarks: What can be done to help premeds?
	
\newpage
\medskip %Median skip?
\nocite{*} %This should go before we print the bibliography
\printbibliography%This only works with biber as the biblatex backend. :/
%\end{workscited}
\end{flushleft}
\end{document}
\}