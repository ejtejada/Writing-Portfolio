%%\cite{Arum-2011}:American colleges are failing entering freshman. Most are lost.

"Although growing proportions of high school graduates are entering higher education, many are not prepared for college-level work and many others have no clear plan for the future. Most American high schools have come to embrace a "college for all" mentality, encouraging students to proceed to higher education regardless of their academic performance" (\cite{Arum-2011} 9).


"An astounding proportion of students are progressing through higher education today without measurable gains in general skills as assessed by the CLA. While they may be acquiring subject- specific knowledge or greater self- awareness on their journeys through college, many students are not improving their skills in critical thinking, complex reasoning, and writing" (\cite{Arum-2011} 14).


%%\cite{Anaya-2001}: The MCAT is a strong indicator of performance in Medical School, but at what cost?
"Investment of their time and energy in decision making is guided by the dream of becoming a doctor. The array of decisions and necessary tasks to meet this goal – selection of an educational program, of the appropriate courses and sequence of courses, of learning strategies and activities, and of extra-curricular activities – indeed requires considerable thought and energy" (\cite{Anaya-2001} 179).

"[P]ublic and private universities typically have the greatest resources: bigger budgets, greater variety of academic programs, and more faculty with Ph.D.s. Thus, universities typically produce greater amounts of research than colleges" (\cite{Anaya-2001} 181).

%%\cite{Anaya-2001}:Money is the trade off needed for an MCAT score.
"There is some evidence in this study that co-curricular activities are negatively associated with student learning as measured by performance on the MCAT. Student involvement in college clubs and organizations (−0.20, p < 0.05) and working for pay (−0.09, p < 0.01) appear to hinder student learning" (\cite{Anaya-2001} 188).

"Careful selection of the type of undergraduate institution, sensible participation in co-curricular activities, enrollment in greater numbers of science courses, and involvement with faculty (research and out-of class) are suggested by the results of this study" (\cite{Anaya-2001} 189).

"This study also pointed to the deleterious effects of increased work hours (for pay), which might possibly be associated with the socio-economic status of the student or with diminished hours available for academic pursuits" (\cite{Anaya-2001} 189).


%%\cite{Cooper-2013} : On the changes to to MCAT 2015 fails to encourage learning chemistry cohesively.
"Unfortunately, the proposed chemistry learning objectives [in the 2015 MCAT] do not seem to provide a coherent framework with which to work. Rather, these objectives are fragments, disconnected ideas, and facts that are clearly parts of existing traditional curricula" (\cite{Cooper-2013} 820).

"The inception of the new medical school requirements could and should provide an incentive to redesign our service courses, and in doing so make them more effective and usefulnot only for premedical students, but for all students. If there is one thing that we do know, it is that current approaches to teaching chemistry are often ineffective" (\cite{Cooper-2013} 820).

"[W]e believe that the structure of traditional chemistry courses makes it even more difficult because the purpose of learning to draw structures is often not immediately and explicitly connected with how the structures are used to determine properties" (\cite{Cooper-2013} 821).

"Students see each step in the process as separate, and are often tested on them separately so that they do not see their overall purpose or value. Such a curriculum does not meet the tenets of meaningful learning, 11,12 in which students must have relevant prior knowledge, new material must be explicitly connected to that prior knowledge, and the student must understand (and accept) the value of the new knowledge" (\cite{Cooper-2013} 821).

%%The circula fails to teach paterns:
"This means that many students, by necessity, must memorize a vast amount of material, rather than understand and predict processes. Although most instructors indicate they teach for understanding (rather than memorization), it is our contention that the very structure of the curriculum makes it quite difficult for students to do this" (\cite{Cooper-2013} 821).

"These courses should not be a hurdle, or a sorting mechanism for medical schools, or a place to stuff disconnected facts and procedures into students’ heads" (\cite{Cooper-2013} 822).


%%\cite{Doering-2015} On hands care for patients build compassion in Pre-Meds
"The pre-med students gained a deep and personal understanding of the importance of being fully present with their patients, and expressed great humility at the limits of their own knowledge" (\cite{Doering-2015} 420).

"[I]ncorporating hospice volunteer programs into pre-medical curriculums around the country would help teach future doctors important lessons about quality patient care" (\cite{Doering-2015} 420).


%%\cite{Horowitz-2010}: On how Premed students select or avoid dificult classes.


"It is somewhat suprising that almost one third of the interviewees (32\%) indicated that their primary reason for being in college was to learn and that they typically sought courses across all subject areas that would challenge them intellectually" (\cite{Horowitz-2010} 225).

"They characterized balance as the taking of risks and experiencing of challenges, while simultaneously not overdoing and/or taking on more than was physically possible. For example, two students acknowledged that occasionally they would sign up for an easier course to make time to devote to a more challenging course" (\cite{Horowitz-2010} 226).


"Only a small number of students (4 of 31) indicated that grades were all that mattered to them and that their only goal in choosing courses was to find courses in which they could easily achieve As" (\cite{Horowitz-2010} 227).

"A total of 3 students reported that an earlier focus of theirs on grades had been influenced and shifted by exposure to peers outside of JC. These students reported that female friends and significant others attending other universities had affected them and had caused them to rethink their goals regarding college" (\cite{Horowitz-2010} 229).

"One student seemed to view mastery and extrinsic goals as somewhat congruous for those courses which were pre-med requirements (introductory sciences and mathematics). He mentioned that doing well in pre-med courses was not sufficient... He explained that his real reason for wanting to learn the pre-med course material was simply in order to do well on the Medical College Admission Test " (\cite{Horowitz-2010} 230).



%\Olsen-2016: On the updated MCAT 2015 and the important of sociology along side bio-chem for premed. Raw science = dehumanization of patients and doctors.

"The updated MCAT appeared in 2015 with four main sections. The first two test the applicants' knowledge and use of concepts in biology [and] chemistry... The second two test the applicants' knowledge, use, and critical analysis of behavior and sociocultural determinants" (\cite{Olsen-2016} 72).

"[P]remedical and medical students learn to distance themselves from the patients with informal stories that dehumanize the cadaver" (\cite{Olsen-2016} 73).

"[T]he sterotype persists not only among non-premeds and faculty, who view premeds as less mature, creative, honest, and humane than non-premeds, but also among premeds themeselves who see themselves as more anxious, less socially active, and less poltiically conscious" (\cite{Olsen-2016} 74).

%but it ain't true
"[E]thnographic data and in-depth interviews with premeds [shows] that dispite the persistence of 'cut-throat' sterotype, premeds in actuality engage in copperative behavior" (\cite{Olsen-2016} 74).

%It's brutal:
"In 2014, there were 47,815 applictions to meical schools, and only 20,343 student marticulated. The average GPA for a marticulated student was 3.69... 60 percent of premeds end up switching from this track, twice the combined attrition rate from all other majors" (\cite{Olsen-2016} 74).

"Sociological instruction is immensely promising for premeds because it provides them with conceptual building blocks and critical perspective to analyze ... the world that they seek to enter: medicine" (\cite{Olsen-2016} 81).

%%Spender-2001: The British Medical Journal's team of editors exposes the rampant cheating in and before medical school, and discusses the ramifications and solutions.

"Of course doctors must have integrity, but is it wrong to assume that a person lacks integrity on the basis of one incident" (\cite{Spencer-2001} 296).

%%Don't publicly hang these students, respect the privacy of academics and leave it to admin:
"It is in itself unethical and unnecessary to wilfully scupper the career of a young doctor .... The doctor in this case has not killed or[sic] poisoned anyone, unlike some of our ilk" (\cite{Spencer-2001} 296).

%NO: Justice was not served with a slap on the wrist.
"Surely the public needs to have confidence that all doctors have shown beyond doubt that theu have the knowledge and ability to practise safe medicine" (\cite{Spencer-2001} 297).

%Failure to expell has cascading harms:
"As a result of this cheating many very able candidates who passed finals honestly will now seem to be less able on paper than some less able students who received the information... Will these dishonest people now be favored for the best senior house officer jobs" (\cite{Spencer-2001} 297)?
