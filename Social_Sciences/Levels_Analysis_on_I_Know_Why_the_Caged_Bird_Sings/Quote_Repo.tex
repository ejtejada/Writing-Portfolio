%% Need from book [8] quotes: 
	%2 quotes for inferential level
	%2 quotes for appreciative level
	%2 quotes for evaluative level
	%2 quotes for critique level
%%AND
%Need from outside articles [2] quotes:
	%2 quotes for essential level(history|literary context)

Literal:
%Maya and Bailey are our two protagonists; they are young black children from LBC.

"Marguerite and Bailey Johnson Jr., from Long Beach, California, en route to Stamps, Arkansas, do Mrs. Annie Henderson" (Angelou 4).

%Maya's grandparents owned and ran a supply store in a plantation town in Arkansas.
"Then she had the Store built in the heart of the Negro area. Over the years it became the lay center of activities in town. On Saturdays, barbers sat their customers in the shade on the porch of the Store, and troubadours on their ceaseless crawlings through the South leaned across its benches and sang" (Angelou 4).

Crique (Tone): 
Self-loathing: Angelou will remind the reader that young Maya does not love herself because she was told by society that whiteness was equivalent to beauty.

Quote:
"I knew that once I put it on I'd look like a movie star. (It was silk and that made up for the awful color.) I was going to look like one of the sweet little white girls who were everybody's dream of what was right with the world" (Angelou 2).
"Wouldn't they be surprised when one day I woke out of my black ugly dream, and my real hair, which was long and blond, would take the place of the kinky mass that Momma wouldn't let me straighten? My light-blue eyes were going to hypnotize them, after all the things they said" (Angelou 3).


%%Unsorted quotes:

%Deep segregation in Arkansas
"In Stamps the segregation was so complete that most Black children didn't really, absolutely know what whites looked like. Other than that they were different, to be dreaded, and in that dread was included the hostility of the powerless against the powerful" (Angelou 13).

"These others, the strange pale creatures that lived in their alien unlife, weren't considered folks. They were whitefolks" (Angelou 13).


"Bailey said the man's things had been cut off and put in his pocket and he had been shot in the head, all because the whitefolks said he did 'it' to a white woman" (Angelou 18).


"If on Judgment Day I were summoned by St. Peter to give testimony to the used-to-be sheriff's act of kindness, I would be unable to say anything in his behalf. His confidence that my uncle and every other Black man who heard of the Klan's coming ride would scurry under their houses to hide in chicken droppings" (Angelou 10).

"Whatever was given by Black people to other Blacks was most probably needed as desperately by the donor as by the receiver. A fact which made the giving or receiving a rich exchange" (Angelou 24).
